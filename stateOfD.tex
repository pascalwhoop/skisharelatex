\subsection{State of digitalization}
We suggest a model to challenge a business to rate itself regarding its "state of digitalization". Companies should objectively rate themselves using their current state and their goal state as a scale and then compare themselves with competitors to create a benchmark. This can describe their current standing in the market and show which components the business already succeeded on and which it needs to focus its attention on. As an example, companies with little previous activity and a low grade of digitalization will start developing a business model and an overall market overview to define their targets while companies that already developed a strategy will skip this step to immediately develop their implementation strategy. With this pattern, a company can develop an individualised strategy based on their industry as well as their individual circumstances.

%To develop a strategy for responding to the impeding changes accompanying \ac{I4.0}, a business should first investigate what components this strategy is composed of.
We suggest to consider three key aspects as core challenges for succeeding during the \ac{DT} which are an adapted form of the four core aspects suggested by the \ac{WEF} \cite{worldforumdigitalenterprise:2016}:
%While the goal of an \ac{I4.0} strategy is apparent, as it is usually congruent with the general economic goals of the business itself, the implementation of this goal is dependent on three fields. 

%We expect everyone to be familiar with the topic of digital transformation so we will refrain from describing estimates of future economic values that range from several hundred billion to a few trillion dollars total. %TODO citation
%We also assume most readers have heard the terms Internet of Things, Industry 4.0 or digital transformation. The argument for why businesses need to carefully evaluate how they implement their strategy and why they execute certain activities can be split into three categories:

\subsubsection{Technology and operations}
The concept of digital transformation suggests that the entire business goes digital. It also implies that companies will have to adapt to a range of new technologies %TODO cite world http://reports.weforum.org/digital-transformation-of-industries/an-introduction-to-the-digital-transformation-of-industries-initiative/
 each of them potentially affecting an organizations core business. These technologies are not yet standardized nor established and some might be considered obsolete again in the near future. While businesses need to innovate to ensure they differentiate from their competitors, %source lecture Schrader? 
 they also need a stable core business to reduce risks. Open standards and technologies that have been widely accepted can offer these securities.
 % Instead of evaluating all these things by themselves, businesses should look towards existing organizations in the form of industry consortia and other non-profit structures to follow their guideline. By doing this, businesses can
The business also needs to improve its operational performance, using technology to both improve savings and enable slack for new models and countering disruptive competitors \cite[p.15ff.]{worldforumdigitalenterprise:2016}.

%technology

\subsubsection{Business Models}
A popular term, disruption, describes what technology innovation does to markets. A new technology can lead to a complete overhaul of existing markets and it is usually accompanied by a different business model that catches existing market players off guard \cite{LucasJr200946}. \citeauthor{gassmann:gallen:2013geschaeftsmodelle} suggest that a core challenge for businesses it not only to be a technological leader but also to develop a new business model that challenges existing market participants and helps a business to differentiate its product from the rest. A common issue for successful businesses is the \emph{Innovator's dilemma}. Businesses need to be willing to disrupt even themselves in order to compete \cite{christensen1997innovator, worldforumdigitalenterprise:2016}.

\subsubsection{Culture and Leadership}
\citeauthor{hammer:2015, LucasJr200946} argue a core task of preparation is the adaptation of the organizational culture to enable openness to innovation and adaption to new market environments. Businesses need to be "results-oriented", allow their employees to have the "freedom to increase innovation"\cite{hammer:2015} and avoid a culture that \emph{"promotes hierarchy and maintain[s] the status quo [because it] will be resistant to disruptive technologies"} \cite{LucasJr200946}.
%business models
%culture
%disruption