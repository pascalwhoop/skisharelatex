\section{Methodology}

To answer our research question we approached the three mentioned aspects of the question systematically. First we performed a literature review, looking for frameworks, guidelines and other forms of guidance for companies looking to ready themselves for the \ac{I4.0}. We selected those that we deemed most promising based on a qualitative review. Initiatives that involve a large amount of global enterprises were considered more promising than frameworks developed by a single business or research team, simply because they have a bigger global influence. From now on we will call these resources simply "framework" as they represent such a structure.

Secondly, to ensure considering all industries in our evaluation, we used the \ac{ISIC} developed by the United Nations to start our filtering of frameworks, mapping the globally defined 21 sectors of industries \cite{ISIC:2008} to those that were found in the literature review. Where possible, we match an industry to a framework that is directly applicable, otherwise evaluating whether a frameworks recommendations can be applied also to the industry that was not specifically mentioned in the literature reviewed. 

Finally we used a model <FILL MODEL NAME @OTT> by <AUTHOR?> to challenge a business to rate itself regarding its "state of digitalization". Companies with little previous activity and a low grade