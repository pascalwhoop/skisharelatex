\section{Conclusion}

In this paper, we provide a systematic approach to make industry 4.0 strategies, concepts and processes applicable in selected \ac{ISIC}-aggregated industries. The opportunities of the implementation of those measures are depending on industry characteristics, the state of digitalization and ultimately the individual \ac{I4.0} Maturity. We found, that \ac{I4.0} frameworks are mainly focusing on one core industry: \emph{Manufacturing}. Therefore, insights of that specific industry may not be applicable in other industries with appreciably different business environment and attributes. Nevertheless, reference models like the \ac{RAMI} and guidelines of the \ac{IIC} are building keystones for every industry and can be used for the primal strategic subsumption. Extending this with the \ac{II4.0MM}, organisations can get profound insights and convertible concepts how they can implement \ac{I4.0} in their respective environment and context.

As this \emph{economic revolution} touches a lot of business areas and interrelations, \ac{I4.0} is a macroeconomic phenomenon which affects whole business models and markets, as well as potent strategic decisions. It is important to get a holistic insight to arrange profound decision-making.

One result of this paper, the \ac{II4.0MM}, is limited to be used as a systematic approach to develop organisational principles and the perception of change in several industries or organisations itself. To develop tangible action plans we recommend further investigation and research, as well as professional consulting, innovation management or interim management.

Some industries are facing an urgent demand for action and others can gradually develop new concepts for the respective change management. This paper shows, that every industry is confronting changes, challenges and chances by the impacts of \ac{I4.0} and needs a strategic concept to build sustainable solutions.

To conclude, we can say that the barriers between industries are getting smaller and the interconnection requires technical standards and cultural change in organisations. Several frameworks can be used to implement \ac{I4.0} measures, individually fitted to the respective business environment. Nevertheless, it is important to state out, that the basis of change and transformation can only be provided by the culture or rather the leadership of organisations. The review of several studies show, that it is important to conduct more studies on the systematic approach which actions are recommended after defining the \ac{I4.0} maturity level or digital readiness combined with the comparison with competitors or new evolving business models. A study, which bridges the \ac{II4.0MM} with real-time data of relevant progress of other organisations or competitors in the respective industry or even the recommendation of cooperation partners, who can leverage the progress of individual \ac{I4.0} implementation, is significantly precious.