\section{Conclusion}

In this paper, we provide a systematic approach to make industry 4.0 strategies, concepts and processes applicable in selected \ac{ISIC}-aggregated sections. The opportunities of the implementation of those measures are depending on business characteristics, the state of digitalization and ultimately the individual \ac{I4.0} maturity. We found, that \ac{I4.0} frameworks are mainly focusing on one core industry: \emph{Manufacturing}. Therefore, insights of that specific industry may not be applicable in other industries with appreciably different business environment and attributes. Nevertheless, reference models like the \ac{RAMI} and guidelines of the \ac{IIC} are building keystones for every industry and can be used for the digital strategy derivation with appropriate transferring of concepts to the actual business and its environment. Extending this with the \ac{II4.0MM}, organizations can get profound insights and convertible concepts how they can implement \ac{I4.0} in their respective environment and context.

As this \emph{revolution} touches a lot of business areas and interrelations, \ac{I4.0} is a macroeconomic phenomenon which affects whole business models and markets, as well as extensive strategic decisions. It is important to get a holistic insight to arrange profound decision-making.

One result of this paper, the \ac{II4.0MM}, is limited to be used as a systematic approach to develop strategic schemes and the perception of change in one specific section or organization itself. Another limitation of this paper is anchored in the holistic review of section-specific literature and frameworks. Therefore, in opposite of detailed information in one specific industry the overview on every \ac{ISIC} section is preferred. This approach limits the in-depth insight of the application of frameworks in one respective section.

To develop tangible action plans we recommend further investigation and research, as well as professional consulting, innovation management or interim management. The interconnection between industries and different business models requires technical standards and cultural change in organizations. Several frameworks can be used to implement \ac{I4.0} measures, individually fitted to the respective business environment. Nevertheless, it is important to state out, that the basis of change and transformation can only be provided by the culture or rather the leadership of organizations. The review of several studies show, that it is important to conduct more studies which recommended actions after defining the \ac{I4.0} maturity level or digital readiness combined with the comparison with competitors or new evolving business models. A study, which investigates a bridge between the \ac{II4.0MM} and reliable data of competitors in the respective industry or even the recommendation of cooperation partners, who can leverage the progress of individual \ac{I4.0} implementation, would fill the gap of knowledge stated out in this paper.

Some industries are facing an urgent demand for action and others can gradually develop new concepts for the respective change management. This paper shows, that many industries are being confronted by changes, challenges and chances by the impacts of \ac{I4.0} and need holistic strategies to build sustainable solutions.


