\section{Terminology}
Before we start however, we would like to define a short list of terms used and what they represent. This is important because of the use sometimes mixed use of many different terms to describe similar concepts.



\begin{itemize}
  \item \textbf{Digital Transformation:} A term that is being described as \emph{'the use of technology to radically improve performance or reach of enterprises'}\cite{westerman2011digital} by the MIT Center for Digital Business can better be put in context by describing it as \emph{'the global accelerated process of technical adaptation by individuals, businesses, societies and nations, which comes as a result of digitalization'}\cite{bonnect2014leading,khan-digital:2016}.

  \item \textbf{Digitalization:} Gartner describes it as the \emph{'use of digital technologies to change a business model and provide new revenue and value-producing opportunities; it is the process of moving to a digital business'} while \citeauthor{khan-digital:2016} simply calls it the \emph{'process of information conversion from the physical to the digital plane'}. The term \emph{digitization} is sometimes used as a synonym to digitalization but we will stick to the former as suggested by \citeauthor{khan-digital:2016}.

  \item \textbf{Internet of Things:} Citing \citeauthor{iot-def:2016}, the \ac{IoT} is \emph{'the network of physical objects that contain embedded technology to communicate and sense or interact with their internal states or the external environment'}.

  \item \textbf{Industry:} An industry is \emph{'defined as the set of all production units engaged primarily in the same or similar kinds of productive activity'}\cite{ISIC:2008}. This is important to differentiate from the German \emph{'Industrie'} which is largely equivalent to the more specific \emph{manufacturing industry}.


  \item  \textbf{Industry 4.0:} Is derived from the German \emph{Industrie 4.0} and describes the German initiative with the same name. It is also sometimes used as a synonym for digitalization in the manufacturing sector \cite{McKinseydigitizationIndustrialSector:2015}, however mostly in the German speaking regions.

  \item  \textbf{Smart factory:} A subtype concept of \ac{DT}, especially focusing on factories.
\end{itemize}
