\subsection{Terminology}
In this subchapter some terms will be defined. This is important because of the mixed use of many different terms to describe similar concepts in both literature and colloquially by professionals.

\begin{itemize}

\item \textbf{Internet of Things:} Citing \citeauthor{iot-def:2016}, the \ac{IoT} is \emph{"the network of physical objects that contain embedded technology to communicate and sense or interact with their internal states or the external environment"}.


  \item \textbf{\acl{CPS}:} The term \ac{CPS} refers to the \emph{"tight conjoining of and coordination between computational and physical resources."}. They \emph{"far exceed those of today in terms of adaptability, autonomy, efficiency, functionality, reliability, safety, and usability"}. They are expected to \emph{"provide large-scale, distributed coordination (e.g., automated traffic control), [be] highly efficient (e.g., zero-net energy buildings), augment human capabilities, and enhance societal well being"}  \cite{cps:nsf:2011}

  \item \textbf{Cloud Computing} is a \emph{"model for enabling ubiquitous, convenient, on-demand network access to a shared pool of configurable computing resources that can be rapidly provisioned and released with minimal management effort or service provider interaction"}\cite{Mell:2011:SND:2206223}.

  \item \textbf{Digitalization:} \citeauthor{khan-digital:2016} simply calls it the \emph{"process of information conversion from the physical to the digital plane"}. Gartner however extends this by describing it as the \emph{"use of digital technologies to change a business model and provide new revenue and value-producing opportunities; it is the process of moving to a digital business"}. The term \emph{digitization} is sometimes used as a synonym to digitalization but we will stick to the former as suggested by \citeauthor{khan-digital:2016}.

  \item \textbf{Digital Transformation:} A term that is being described as \emph{"the use of technology to radically improve performance or reach of enterprises"}\cite{westerman2011digital} by the MIT Center for Digital Business can better be put in context by describing it as \emph{"the global accelerated process of technical adaptation by individuals, businesses, societies and nations, which comes as a result of digitalization"}\cite{bonnect2014leading,khan-digital:2016}.

  \item  \textbf{Industry 4.0:} Is derived from the German \emph{Industrie 4.0} and describes the German initiative with the same name. It is also sometimes used as a synonym for digitalization in the manufacturing section \cite{McKinseydigitizationIndustrialSector:2015}, however mostly in the German speaking regions. We consider it as the \acl{DT} of the manufacturing section.


  \item \textbf{Industry:} An industry is \emph{"defined as the set of all production units engaged primarily in the same or similar kinds of productive activity"}\cite{ISIC:2008}. This is important to differentiate from the German \emph{"Industrie"} which is largely equivalent to the more specific \emph{manufacturing industry}. We will, to avoid confusion with the term industry 4.0 and to follow a systematic approach, use the term 'section' to describe groups of industries. The term has a more abstract connotation and has also been used by \ac{UN} in \ac{ISIC}\cite{ISIC:2008}.
%TODO !!!!!!! replace all 'industry' or 'section' with 'section'

\item \textbf{Section} The \acf{ISIC} defines sections as  \emph{"categories intended to facilitate economic analysis"}\cite{ISIC:2008}. It is the highest level of categories, subdivided into \emph{divisions, groups and classes}, offering a four layer classification scheme of all economic activities independent of country or type. 

\end{itemize}

\ac{IoT}, \ac{I4.0}, \ac{CPS} and Cloud Computing can be considered initiators or enabling technolgies that lead to an increase of digitalization in business and society. This increased digitalization, which can be considered a continuous process, now causes a sudden surge in adaptation and investments by businesses and consumers alike into these technologies that have such wide effects on all aspects of society that it is considered a transformation or revolution \cite{Kagermann:2013}.
Figure \ref{fig:terms} summarizes the different terms and their interrelation.

\begin{figure}[H]
\centering
\begin{tikzpicture}[node distance=2cm]

%first row
\node (IoT) [process] {IoT};
\node (CPS) [process, right of=IoT, xshift=0cm] {CPS};
\node (CC) [process, right of=CPS, xshift=1cm] {Cloud Computing};
%second,third row
\node (D) [process, below of=CPS] {Digitalization};
\node (DT) [process, below of=D, xshift=-1cm] {Digital Transformation};
\node (I40) [process, right of=DT, xshift=1cm] {Industry 4.0};

%arrows
\draw [arrow] (IoT) -- (D);
\draw [arrow] (CPS) -- node {lead to}  (D);
\draw [arrow] (CC) -- (D);
\draw [arrow] (D) -- node [xshift=0.7cm]{triggers}(DT);
\draw [arrow] (D) -- (I40);


\end{tikzpicture}
\caption{Relation between terms} \label{fig:terms}
\end{figure}

%JO
%TODO 1) Framework als Begriff klarstellen? 2) Maturity Model mit aufnehmen sobald wir dieses festgelegt haben
