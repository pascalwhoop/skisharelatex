\section{Introduction}\label{Introduction}
% what is Digital Transformation and why is it being proposed to happen
\ac{DT}, the \ac{IoT}, Smart Factories,
\ac{I4.0}. Many words describing a vague ex ante declared
\emph{revolution} of the economy that many scientists and professionals are discussing.
% Gartner prophesizes over 6 billion IoT devices in use by the end
%of 2016 with over 13 billion expected by 2020
%\cite{gartner-iot-number-devices}.
There is a big interest from the public and private sectors with many big consulting companies offering services to their customers regarding strategy, implementations, project management and technological support.
\cite{westerman2011digital,mckinsey-nine-questions,bcg-dt,accenture-dt:2015}.
Nonprofit organizations and governments are also organizing initiatives and associations like the German governments \emph{\ac{I4.0Init}} \cite{i40-web} initiative or the \emph{\ac{IIC}} \cite{iic-web} based in the United States.
%- most are focused on 3-4 industries

However, most are focusing on a few core industries, not following a systematic approach of evaluating the effect of the impeding changes on all sectors \cite{westerman2011digital, pwc2016survey}. 
%- many industries, although interesting, are left out or just touched in single, specialised articles
Many industries, e.g. construction, although interesting, are often left out or considered only in separate specialised sources \cite{rolandbergerBauwirtschaft:2016}. 
%- big initiatives by governments or corporate associations try to cover all industries but still, they are missing some
Even the nation wide initiative \acl{I4.0Init} is specialised and focuses only on a subset of the whole variety of sectors \cite{umsetzungsstrategie:2015}. 
%- a formal overview of all sectors and a mapping of guidelines for each is needed to get an overview of the true cross-sector-spanning nature of the \ac{DT}

A complete list of all sectors has been developed by the United Nations in the \ac{ISIC}\cite{ISIC:2008}. A formal overview including all sectors defined in this empirically based paper can ensure the complete coverage of all sectors regarding their potential for \ac{I4.0} and \ac{DT}. The research question therefore is the following:
%- RQ: What generic and specialised guidelines exist for all sectors regarding the implementation of an \acf{I4.0} strategy?

\emph{What guidelines exist for each sector regarding the implementation of an \acf{I4.0} strategy?}


This question consists of three key aspects that will be put together to create a structured strategic process recommendation:
\begin{itemize}
    \item \textbf{Find frameworks}, initiatives and consortia adequate for a wide variety of businesses offering a holistic view on \ac{I4.0} and \ac{DT}
    \item \textbf{Evaluate each sector} to find suiting frameworks
    \item Individualize recommendations based on a \textbf{companies state of digitalization}.
\end{itemize}

The expected result is a systematic overview of all sectors and frameworks that can be used as guidelines for developing \ac{I4.0} strategies independent of a companies sector or size. 
% what implications are foreseen and what does this mean for organizations
%====SNIPPET TAKE ====
%A wide range of industries are affected \cite{westerman2011digital,iic-web}. Companies like Uber and Tesla are intentionally pushing into the market of autonomous driving \cite{uber-autonomous,tesla-autonomous-blog}, a facet of the digital transformation, but virtually any company can get involved, even if they did not intend to as a recent case of DDOS cyber attacks showed which disabled big parts of the U.S. internet connectivity and left thousands of companies out of touch with their customers. The attackers used a large number of hacked IP security cameras which reminded organizations about a core challenge, IT security \cite{forbes-ddos-cameras}, as well as the fact that any company that is currently doing business will have to adapt to the digital transformation of our economy.

% What guidelines are already existing

%====SNIPPET TAKE ====
%With a variety of organizations and businesses showing interest in the development, the need for standardization and common structures becomes inherent to increase compatibility, avoid duplicate efforts and increase positive network effects. The German \ac{I4.0} initiative for example developed a \emph{\ac{RAMI}} for companies to use as a base of classifying standards and coordinating efforts. 
%who is left out ?
%This initiative however has a strong focus on manufacturing industries, a focus that has been apparent throughout our literature review. Other industries such as "Agriculture, Construction, Financial Services or Wholesale and retail trade" are seldom the focus \cite{bauer2015industrie,barometer:2016}. 

%====SNIPPET TAKE ====
%While the market is buzzing and everyone is talking about the expected changes, strategic business decisions, especially those that commit significant resources and create technological or economical dependencies, should be evaluated thoroughly to ensure success of the business.

%Due to the variety of frameworks, white papers and consulting offers found in our research as well as the imprecise use of many terms and concepts by most sources we want to create a guideline for managers intending to ready their organization for the \acl{I4.0}, independent of their industry along the following question:

%research question
%\emph{How can organizations strategically approach Industry 4.0, specific to their organizational environment and industry, integrating both their own previous efforts as well as global initiatives and frameworks to receive individual implementation recommendations.}




%JO
%TODO DONE 1) "many professionals are buzzing about" finde ich noch nicht gut formuliert. Vorschlag: "many scientists and professionals are discussing about" 
%TODO DONE 2) Japan IVI ausschließen? Haben wir ja nicht weiter betrachtet. 

\subsection{Methodology}

To answer our research question we approached the three mentioned aspects of the question systematically. First we performed a literature review, looking for frameworks, guidelines and other forms of guidance for companies looking to ready themselves for the \ac{I4.0}. 
%TODO how many did we look at? and which did we choose? literature review statistics
We selected those that had a holistic view on many sectors or that covered those sectors that were usually left out by most others. Initiatives that involve a large amount of global enterprises were considered more promising than frameworks developed by a single business or research team, simply because they have a bigger global influence. From now on we will call these resources simply "framework" as they represent such a structure.

Secondly, to ensure considering all relevant industries in our evaluation, we used the \ac{ISIC} developed by the United Nations to start our filtering of frameworks, mapping the aggregated, globally defined 11 sectors of industries \cite{ISIC:2008} to those that were found in the literature review. Where possible, we match an industry to a framework that is directly applicable, otherwise evaluating whether a frameworks recommendations can be applied also to the industry that was not specifically mentioned in the literature reviewed. 

Finally we suggest an industry-focused, applicable model for businesses to rate themselves as well as their competition, so they can compare their current maturity state with a target state of digitalization. We focus on three categories: technology \& operations, business model and culture, as suggested by the \ac{WEF} whitepaper \cite{worldforumdigitalenterprise:2016}.


