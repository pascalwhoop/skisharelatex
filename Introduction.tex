\section{Introduction}\label{Introduction}
% what is Digital Transformation and why is it being proposed to happen
\ac{DT}, the \ac{IoT}, Smart Factories,
\ac{I4.0}. Many words describing a vague ex ante declared
\emph{revolution} of the economy that many professionals are buzzing
about. Gartner prophesizes over 6 billion IoT devices in use by the end
of 2016 with over 13 billion expected by 2020
\cite{gartner-iot-number-devices}. Governments and companies alike are
investing heavily into the anticipated technology changes with efforts
like the German governments \emph{Industrie 4.0} \cite{i40-web} initiative %, the \emph{Industrial Value Chain Initiative} (IVI) \cite{ivi-web} originating in Japan
or the \emph{\ac{IIC}} \cite{iic-web} based in the United States.
There is also a big interest from the private market with many big consulting companies offering services to their customers regarding strategy, implementations, project management and technological support
\cite{westerman2011digital,mckinsey-nine-questions,bcg-dt,accenture-dt:2015}.

% what implications are foreseen and what does this mean for organizations
A wide range of industries are affected \cite{westerman2011digital,iic-web}. Companies like Uber and Tesla are intentionally pushing into the market of autonomous driving \cite{uber-autonomous,tesla-autonomous-blog}, a facet of the digital transformation, but virtually any company can get involved, even if they did not intend to as a recent case of DDOS cyber attacks showed which disabled big parts of the U.S. internet connectivity and left thousands of companies out of touch with their customers. The attackers used a large number of hacked IP security cameras which reminded organizations about a core challenge, IT security \cite{forbes-ddos-cameras}, as well as the fact that any company that is currently doing business will have to adapt to the digital transformation of our economy.

% What guidelines are already existing
With a variety of organizations and businesses showing interest in the development, the need for standardization and common structures becomes inherent to increase compatibility, avoid duplicate efforts and increase positive network effects. The German \ac{I4.0} initiative for example developed a \emph{\ac{RAMI}} for companies to use as a base of classifying standards and coordinating efforts. 
%who is left out ?
This initiative however has a strong focus on manufacturing industries, a focus that has been apparent throughout our literature review. Other industries such as "Construction, agriculture, forestry or fishing" are seldom the focus \cite{bauer2015industrie,barometer:2016}. 

While the market is buzzing and everyone is talking about the expected changes, strategic business decisions, especially those that commit significant resources and create technological or economical dependencies, should be evaluated thoroughly to ensure success of the business.

Due to the variety of frameworks, white papers and consulting offers found in our research as well as the imprecise use of many terms and concepts by most sources we want to create a guideline for managers intending to ready their organization for the \acl{I4.0}, independent of their industry along the following question:

%research question
\emph{How can organizations strategically approach Industry 4.0, specific to their organizational environment and industry, integrating both their own previous efforts as well as global initiatives and frameworks to receive individual implementation recommendations.}

This question consists of three key aspects that will be put together to create a structured strategic process recommendation:
\begin{itemize}
    \item \textbf{Find frameworks}, initiatives and consortia adequate for a wide variety of businesses offering a holistic view on \ac{I4.0} and \ac{DT}
    \item \textbf{Evaluate each industry} to find suiting frameworks
    \item Individualize recommendations based on a \textbf{companies state of development}.
\end{itemize}

%\subsection{Research and Outcomes}
