\section{Individual I4.0 Maturity Model}
\label{sec:i4mm}

To enable an individual \ac{I4.0} strategy and applicable guidance, the section-specific frameworks provide general information in the respective section to achieve essential concepts. One key element is missing to make these concepts lead to an individual strategy to get \ac{I4.0} processes implemented in the organization of the reader of this paper: The competitive environment in this specific industry combined with the individual state of digitalization. To solve this, we gleaned insights to outline the \ac{II4.0MM}. The objective is to receive an individual scheme on a strategic level, industry-specific and connected to the status quo of the considered organization \& competition, to implement relevant and effective \ac{I4.0} processes and ultimately transform to sustainable solutions.

The following maturity model insights are the basis for the \ac{II4.0MM}, but are either only focusing on manufacturing-specific \ac{I4.0} or not aggregating all relevant data

\begin{enumerate}
\item Industry 4.0 Maturity Model \cite{Schumacher2016161}
\item The Connected Enterprise Maturity Model \cite{RockWellAutomation-connectedEnterpriseMaturityModel}
\item The Digital Advantage: How digital Leaders outperform their peers in every industry \cite{CapgeminiMaturityModelDigitalAdvantage}
\end{enumerate}

The \ac{II4.0MM} is structured into 3 steps, which are described subsequently, enriched with the appropriate frameworks and studies.


\begin{figure}[H]
\centering
\includegraphics[width=1\columnwidth]{images/II40MM_grafik.PNG}
\caption{\ac{II4.0MM} 3 steps: State of digitalization, Self-Assessment (DRA), Competitor Benchmark}
\label{fig:II4.0MM}
\end{figure}

It is important to define what is meant with the term \emph{\ac{II4.0MM}}. In this paper, we combine the three key aspects of the state of digitalization, which are described in chapter \emph{2.2},
with self-assessment tools to get the benchmark of the competition in the considered industry to shape an applicable model. This can be used to get strategic leaders converted to implement digital and \ac{IoT} processes in their respective organization.
Other studies may call it \emph{digital Readiness}, which is not focusing on \ac{I4.0} or \ac{DT} in detail.

The first step of the \ac{II4.0MM} is to determine the general type of digital maturity. The method we recommend is delivered by the four Types of Digital Maturity by \citeauthor[p.4]{CapgeminiMaturityModelDigitalAdvantage}:

\begin{figure}[H]
\centering
\includegraphics[width=1\columnwidth]{images/maturityModel_4segments_capgemini.PNG}
\caption{Four Types of Digital Maturity: Fashionistas, Digirati, Beginners, Conservatives from \citeauthor{CapgeminiMaturityModelDigitalAdvantage} \cite[p.4]{CapgeminiMaturityModelDigitalAdvantage}}
\label{fig:FourTypesMaturityCapgemini}
\end{figure}

Those four types are distinguished by the affiliation to two dimensions: digital intensity (technology-enabled initiatives) and transformation management intensity (leadership capability) \cite{CapgeminiMaturityModelDigitalAdvantage}. Extending those dimensions with the business models as the third key aspects of the state of digitalization, we can see an analogy to our predefined model (chapter 2.2). Every organization can classify themselves into one described type by answering specific questionnaires \cite{CapgeminiMaturityModelDigitalAdvantage}.

The second step is to deploy a self-assessment \ac{DRA} \cite{Schumacher2016161} \cite{ReadinessIndustrie40Impulse} \cite{i40-self-assessment-PwC:2016}. The solution of the \ac{DRA} might strengthen the predetermination of the digital maturity type or reveal new chances or challenges regarding the implementation of \ac{I4.0} measures. The \ac{II4.0MM} suggests to use one of the following self-assessment \ac{DRA}:
\begin{enumerate}
\item Digital Readiness Assessment: Does your business strategy work in a digital world? \cite{ey-dra}
\item IMPULS – Industrie 4.0 Readiness \cite{ReadinessIndustrie40Impulse}
\item Industry 4.0 Self Assessment \cite{i40-self-assessment-PwC:2016}
\end{enumerate}

Finally, the \ac{II4.0MM} involves the establishing of a connection to the individual competition of the respective organisation. Step three is the competitor benchmark. General information can be found in chapter 3 \emph{Section specific application of frameworks} and the respective recommended frameworks with detailed information. Extending those with tools like the \ac{BMN}, described in chapter 2.3.4., delivers even more insights.

The holistic evaluation of industry-specific characteristics and the information of the progress of competitors or threatening new business-models are leading to suitable benchmark data. This step concludes the \ac{II4.0MM} and delivers a scheme, which can be used to implement \ac{I4.0} according to the individual context.