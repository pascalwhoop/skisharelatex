\section{Background}


1. I40 and IIC definition
2. koloss industries clustering 
3. what statistcs do we base our categorisation on
4. where do you see yourself model (aka early adopter etc)


\subsection{Summary of trends and guidelines}

Both the industry and the academic world offer a variety of frameworks, concepts, consulting services and guidelines for businesses, institutions and even cities to employ to optimize their utility from the Digital Transformation.
We analyzed the documentations, guidelines and frameworks from the IIC and I4.0 and summarize their common guidelines in the following chapter.

\subsubsection{Industrie 4.0}
The German Industrie 4.0 initiative was created by the German government to improve Germanys economical position in the global market. According to the \emph{Plattform Industrie 4.0}, I4.0 is a specialization of the \emph{Internet of Things and Services} and applies to a subset of all industries, mainly focused on industrial production and manufacturing
\cite[p.41]{umsetzungsstrategie:2015}
 The following list summarizes the core goals. \footnote{It should be noted that only some of these goals are relevant to individual businesses trying to improve their strategy while others are global environmental necessities. Those that require action of individual businesses are marked with an (X).}

\begin{itemize}
	\item  Standardization (X)
	\item  Reducing of complexity (X)
	\item  Wideband infrastructure
	\item  Security (X)
	\item  Work culture and organization (X)
	\item  Education
	\item  Legal constraints (X)
	\item  Efficiency (X)
\end{itemize}
\cite[p.8]{umsetzungsstrategie:2015}

The I4.0 initiative offers several artifacts to support organizations in the transition to a digitalized organization as well as to facilitate the coordination between organizations and industries in researching and developing standards and technologies. One such example is RAMI which has been created as a guideline to avoid definition of multiple, redundant and conflicting standards and communication strategies \cite[p. 41]{umsetzungsstrategie:2015}.

%

\subsubsection{Industrial Internet Consortium}
The IIC is an 'open membership organization with 250 members from 30 countries, formed to accelerate the development, adoption and widespread use of interconnected machines and devices, intelligent analytics, and people at work'\cite{iic-progress:2016}. Its goals are as follows:

\begin{itemize}
	\item  Creation of use cases and testbeds
\item  Develop reference architectures and frameworks
\item  Influence the global development standards process
\item  Facilitate open forums
\item  Build confidence around approaches to security.
\end{itemize}
\cite{iic-aboutus:2016}


\subsubsection{Integrating IIC and I4.0}

Both the IIC and the I4.0 have announced collaboration efforts to ensure their goal of common standards and structures is achievable. While the IIC's efforts are targeted at a higher level of abstraction, the I4.0's focus is more narrow, focusing on the manufacturing and production industry. 


\begin{itemize}
	\item Business Model verification and redesign \cite{gassmann2013geschaeftsmodelle}
	\item Organizational Culture
	\item Technological capabilities and expertise
\end{itemize}




%\subsection{Terminology}

%\begin{itemize}
%\item
%what is Industrie 4.0 in relation to other words in the english-speaking environment
%\item
%does Industrie 4.0 also apply to industries that aren't involved in manufacturing
%\end{itemize}

%Compare to
%\cite{Wirtschaft_en:2016}
%and \cite{Wirtschaft:2016}
%for how the official translation handles the terms
